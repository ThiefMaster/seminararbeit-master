\chapter{Grundlagen}
\pagenumbering{arabic}

\section{Python}

Bei Python handelt es sich um eine dynamische \emph{general-purpose}-Scriptsprache, d.h. eine Sprache
die nicht für einen bestimmten Verwendungszweck wie beispielsweise Webapplikationen entwickelt
wurde. Sie ist für alle verbreiteten Betriebssysteme verfügbar und es existieren verschiedene
Implementierungen; die bekanntesten sind die Referenzimplementierung CPython (C), Jython (Java),
PyPy (Python) und IronPython (C\#, .NET).

Die Sprache legt großen Wert auf lesbaren Code, was insbesondere daran zu erkennen ist, dass Blöcke
nicht wie in den meisten anderen Sprachen durch geschweifte Klammern oder
\emph{begin}/\emph{end}-Statements definiert werden sondern einzig und allein durch die Einrückung.
Dementsprechend führt inkonsistente Einrückung auch zu einer entsprechenden Fehlermeldung.

\begin{lstlisting}[caption=Fehlerhafte Einrückung]
>>> def fail():
...     a = 'b'
...      b = 'c'
  File "<stdin>", line 3
    b = 'c'
    ^
IndentationError: unexpected indent
\end{lstlisting}


\section{Webapplikationen}

Wie bereits zuvor erwähnt ist Python nicht speziell auf das Web zugeschnitten. Allerdings gibt es
neben der Option, einen Python-basierten Webserver einzusetzen diverse Möglichkeiten, Python-Code
über einen bereits vorhandenen Webserver auszuführen. Im Rahmen dieses Kapitels werden nur die
bekanntesten Technologien kurz vorgestellt; neben diesen existieren noch weit verbreitete Lösungen
wie \emph{mod\_wsgi} und das veraltete \emph{mod\_python}, die allerdings den
\emph{Apache}-Webserver voraussetzen.

\subsection{Python-basierte Webserver}

Insbesondere während der Entwicklung bietet sich diese Lösung an, da man dadurch extrem flexibel
ist und beispielsweise webbasierte Debugger nutzen kann und Informationen direkt auf der Shell, in
der der Webserver läuft, ausgeben kann.

Für Produktivsysteme ist diese Option allerdings nicht optimal, da oftmals bereits ein klassischer
Webserver wie \emph{Apache} oder \emph{nginx} läuft und man nach Möglichkeit nicht mehrere Webserver
auf demselben System betreiben will - dies würde mehrere IP-Adressen bzw. Reverse Proxying
voraussetzen. Darüberhinaus gibt es sinnvolle Gründe, Webserver und Python-Interpreter voneinander
zu trennen - beispielsweise um letzteren unter einem separaten Benutzeraccount laufen zu lassen.

\subsection{CGI}

Das Common Gateway Interface, kurz CGI\footnote{\href{http://www.ietf.org/rfc/rfc3875}{RFC 3875}
\citep{rfc3875}},
ist der wohl bekannteste und älteste Standard, um externe Programme über einen Webserver
auszuführen.

Dabei wird für jeden Aufruf einer entsprechenden URL ein neuer Prozess gestartet; Metadaten wie die
IP-Adresse des Clients oder die abgerufene URL werden via Umgebungsvariablen weitergegeben. Die
Standarddatenströme \emph{stdin} und \emph{stdout} dienen der Übergabe von Request-Bodies
(beispielsweise bei \emph{HTTP POST}) bzw. dazu, die Antwort des Programms an den Webserver
zu übergeben.

CGI ist daher sehr einfach zu implementieren. Python bietet mit dem
\href{http://docs.python.org/library/cgi.html}{\lstinline{cgi}-Modul}\footnote{\citep{pycgi}}
Hilfsfunktionen an, die die
Kommunikation über das Common Gateway Interface abstrahieren und beispielsweise Formulardaten
parsen können sodass man komfortabel auf sie zugreifen kann. Allerdings muss für jeden Request ein
neuer Prozess gestartet und jeglicher Initialisierungscode erneut ausgeführt werden. Dieser Overhead
ist außer bei sehr einfachen Anwendungen mit wenigen Benutzern problematisch und selbst dort würde
er zu unangenehm langen Ladezeiten führen wenn beispielsweise ein ORM-Framework initialisiert werden
müsste.

\autoref{lst:python-cgi} zeigt ein kleines Beispielscript welches einen GET-Parameter bzw. einen
Standardtext ausgibt. Man erkennt an dem manuell ausgegebenen \emph{Content-type}-Header gut, dass
CGI sehr simpel ist und das HTTP-Protokoll nur minimalst abstrahiert.

\begin{lstlisting}[caption=Python-CGI-Script,label=lst:python-cgi]
#!/usr/bin/env python

import cgi

form = cgi.FieldStorage()
print 'Content-type: text/plain'
print
print 'Hello %s' % form.getfirst('name', 'World')
\end{lstlisting}

Bei der Verwendung von CGI entsprechen Teile der URL normalerweise gleichnamigen Pfaden im
Dateisystem. So würde eine Anfrage nach \emph{/foo/bar.py} mit hoher Wahrscheinlichkeit von einer
Datei namens \emph{bar.py} im Ordner \emph{foo} innerhalb des Document Roots, d.h. des obersten über
HTTP erreichbaren Ordners verarbeitet werden. Im Falle einer Fehlkonfiguration des Webservers würde
die Datei nicht ausgeführt werden sondern der Benutzer würde den Quellcode dieser Datei sehen -
einschließlich möglicherweise enthaltener Datenbankzugangsdaten.

\subsection{FastCGI}

Ziel von FastCGI ist es, den Overhead von CGI zu vermeiden, indem nicht für jeden Request ein neuer
Prozess gestartet wird sondern mehrere Prozesse dauerhaft laufen und der Webserver alle Anfragen für
entsprechende URLs über ein Socket an diese Prozesse weitergibt.

Dadurch ist es auch nicht mehr notwendig, die Applikation jedes Mal neu zu initialisieren, und der
Python-Interpreter kann in einem separaten Prozess laufen der durchaus einen anderen Besitzer haben
kann als der Webserver selbst.

Python selbst unterstützt kein FastCGI, allerdings existieren diverse Drittanbieter-Module, die
eine WSGI-Applikation über FastCGI bereitstellen.

\subsection{WSGI}

Bei WSGI\footnote{\href{http://www.python.org/dev/peps/pep-0333/}{Web Server Gateway
Interface} \citep{wsgi}}
handelt es sich um eine Python-spezifische Kommunikationsschnitstelle zwischen Python-Applikationen
und Webservern.

Wie schon CGI und FastCGI stellt auch WSGI eine standardisierte Schnittstelle zur Verfügung, d.h.
eine Webapplikation die WSGI unterstützt kann in jedem beliebigen Webserver verwendet werden, der
WSGI unterstützt.

\begin{lstlisting}[caption=Einfache WSGI-Applikation,label=lst:wsgi-app]
def application(environ, start_response):
    start_response('200 OK', [('Content-Type', 'text/plain')])
    yield 'Hello World!'
\end{lstlisting}

Erwähnenswert ist dass anders als bei CGI in der Regel keine 1:1-Zuordnung zwischen URLs und Dateien
existiert. Eine WSGI-Applikation wird normalerweise innerhalb eines Ordners (oftmals \emph{/})
"gemounted", sodass der Webserver alle Anfragen die in diesen Ordner führen sofort an die
WSGI-Applikation weitergibt. Diese ordnet sie dann meist einer bestimmten Funktion oder Klasse zu.
Allerdings muss sich dazu keine einzige Python-Datei innerhalb des Document Roots befinden - im
Falle einer Fehlkonfiguration würde der Benutzer also schlimmstenfalls ein leeres Verzeichnislisting
oder eine \emph{File Not Found}-Fehlermeldung erhalten.

Ein weiteres Feature von WSGI ist die Unterstützung von \emph{Middleware}. Dabei handelt es sich um
Schichten die sich einfach um eine existierende Applikation legen lassen und den Datenstrom
zwischen der über- und untergeordneten Schicht beeinflussen können. Dies kann beispielsweise dazu
genutzt werden um Anfragen an bestimmte URLs abzufangen und anderweitig zu behandeln. Die Middleware
in \autoref{lst:wsgi-middleware} fängt beispielsweise alle Anfragen an Dateien innerhalb des Pfades
\emph{/shared} (relativ zum Mountpoint der WSGI-Applikation) ab und mappt selbige auf einen real
existierenden Ordner im Dateisystem, der statische Dateien enthält.

\begin{lstlisting}[caption=WSGI-Middleware,label=lst:wsgi-middleware]
application = SharedDataMiddleware(application, {
    '/shared': os.path.join(os.path.dirname(__file__),
                            'shared')
})
\end{lstlisting}


\section{Webframeworks}

Wie in \autoref{lst:wsgi-app} zu sehen ist, ist es sehr einfach eine WSGI-Applikation zu schreiben,
ohne zusätzliche Module zu verwenden. Allerdings möchte man in der Regel URLs auf Funktionen/Klassen
mappen, Formulardaten einschließlich Dateiuploads verarbeiten und auch gewisse Standard-Features wie
beispielsweise Sessions, Templating und Datenbankzugriff komfortabel nutzen können ohne die
entsprechenden Lowlevel-APIs nutzen zu müssen.

An dieser Stelle kommen Webframeworks ins Spiel. Sie abstrahieren das WSGI-Interface und bereiten
die Daten eines HTTP-Requests entsprechend auf, bevor sie die eigentliche Applikation erreichen.
Dadurch kann man sich als Entwickler in den meisten Fällen darauf konzentrieren, die eigentliche
Applikationslogik zu schreiben. Für Sonderfälle wie beispielsweise Streaming von größeren Dateien
ist oftmals zusätzlicher Code notwndig, allerdings bietet ein gutes Framework entsprechende
Schnittstellen an, sodass man den Code des Frameworks selbst normalerweise niemals modifizieren
muss.

Bei der Wahl eines Frameworks ist darauf zu achten, dass es den Anforderungen der Applikation
entspricht, die man damit entwickeln will. So sind viele Frameworks beispielsweise für
CRUD\footnote{\emph{Create, Read, Update, Delete}} optimiert, da dieser Workflow in vielen
Anwendungen häufig verwendet wird. Ein gutes Beispiel für eine solche Applikation ist ein einfaches
Blog. Der Administrator kann dort neue Posts erstellen (\emph{Create}), Posts auflisten
(\emph{Read}), Posts verändern (\emph{Update}) und auch löschen (\emph{Delete}). Für alle diese
Operationen kann oftmals ein nach \emph{Schema F} aufgebautes Benutzerinterface verwendet werden,
sodass es sich in einem entsprechenden Framework anbietet selbiges automatisch zu generieren - der
Entwickler muss dazu oftmals nur die entsprechenden Datenbanktabellen bzw. Mapping-Objekte angeben.
Mit entsprechenden Hooks kann ein Framework dem Entwickler auch ermöglichen, in den verschiedenen
Operationen benutzerdefinierten Code auszuführen - nach einer erfolgreichen \emph{Create}-Operation
würde es sich beispielsweise anbieten, den Link zum gerade erstellten Post via Twitter zu
verbreiten.
