\chapter{Ausblick}

\section{Verbreitung}
Flask wird in einer Vielzahl von Projekten eingesetzt. Während es sich dabei größtenteils um private
Websites oder kleinere Open-Source-Projekte handelt wird Flask auch von einigen Firmen eingesetzt.

Die bekannteste Firma, die Flask einsetzt, dürfte \emph{DISQUS} sein. Bei \emph{disqus.com} handelt
es sich um einen Onlinedienst, der es Websitebetreibern ermöglicht, ohne großen Aufwand eine
Kommentarfunktion mit verschachtelten Diskussionen zu integrieren. Derzeit wird Disqus laut Anbieter
von ca. 91 Millionen Benutzern verwendet und ist auf ca. 1.8 Millionen Websites integriert, bei
denen man auch Größen wie CNN und Bloomberg findet.

\img{disqus-logo-dark.png}{100px}{Disqus-Logo}{Disqus-Logo}

Selbstverständlich wird Flask noch auf deutlich mehr Seiten eingesetzt - da bei der Benutzung von
Flask jedoch kein \enquote{Powered by Flask}-artiger Hinweis eingebaut werden muss, ist es in der
Regel für einen Außenstehenden nicht zu erkennen, dass eine Website Flask-basiert ist. Es gibt
jedoch auf der Flask-Website eine
\href{http://flask.pocoo.org/community/poweredby/}{Liste}\footnote{\citep{poweredbyflask}}, in der
sich jeder Benutzer von Flask eintragen kann. Diese ist auch für angehende Flask-Entwickler
nützlich, da oftmals auch der Quellcode der Seiten verlinkt ist und man somit bereits vorhandenen
Flask-Code anschauen kann.

\section{Alternativen}
Das wohl bekannteste Python-Webframework ist
\emph{Django}\footnote{\href{https://www.djangoproject.com/}{https://www.djangoproject.com/}
\citep{django}}; es
ist primär auf \emph{Rapid Application Development} ausgelegt und wird daher von den Entwicklern
auch mit \enquote{for perfectionists with deadlines} bewerben. Anders als Flask enthält es alles, was
für eine typische datenbankgestützte Webanwendung notwendig ist. Ein komplettes ORM-System ist
bereits integriert und Caching und i18n\footnote{Internationalization} sind ebenfalls Teil von
Django. Die Django-Template-Engine hat starke Ähnlichkeit mit dem in Flask genutzten Jinja2;
allerdings kann letzteres auch standalone, d.h. ohne dass Flask installiert ist, benutzt werden. Das
URL-Routing-System in Django besitzt eine gewisse Ähnlichkeit mit dem von Flask, allerdings setzt es
auf reguläre Ausdrücke und das Python-Modul und die Python-Funktion die für die URL zuständig ist
wird als String der Form \lstinline{'package.subpackage.modulename.function'} übergeben. Dies hat
den Vorteil, dass alle URLs an einer zentralen Stelle definiert werden statt via Decorator direkt
bei der dazugehörigen Funktion, allerdings ist letzteres gerade bei kleineren Anwendungen
komfortabler und auch Flask bietet die Möglichkeit, URL-Routes an einer beliebigen anderen Stelle im
Code zu definieren. Ein weiterer erwähnenswerter Unterschied zu Flask ist, dass man bei Django
selbst bei einer kleinen Anwendung nicht mit einer einzelnen Datei auskommt sondern auf eine gewisse
Datei- und Ordnerstruktur angeweisen ist. Diese kann jedoch mit dem zu Django gehörenden
\emph{django-admin.py}-Script automatisch erzeugt werden. Der wohl größte Vorteil von Django ist,
dass es \emph{DAS} Python-Webframework schlechthin ist und es dementsprechend vielen fertigen Code
dafür gibt und viele Python-Webentwickler bereits damit vertraut sind. Dadurch, und dass für
\emph{CRUD}-Anwendungen nur sehr wenig eigener Code notwendig ist, bietet sich Django an sofern man
eine solche Anwendung entwickeln möchte oder vermeiden will, dass sich neue Entwickler erst in ein
weniger bekanntes Framework einarbeiten müssen.

Ein weiteres, ebenfalls weit verbreitetes, Webframework ist
\emph{Web2Py}\footnote{\href{http://web2py.com}{http://web2py.com} \citep{web2py}}. Wie schon Django handelt es
sich auch bei Web2Py um ein \emph{full-stack} Framework, welches von Hause aus alles enthält, was
eine typische Anwendung benötigt. Statt eines ORMs enthält es jedoch nur eine
Datenbankabstraktionssschicht, die eine Python-API zur Verfügung stellt, über die man SQL-Abfragen
erstellen kann ohne SQL zu schreiben. Die Templateengine von Web2Py ermöglicht es, beliebigen
Python-Code in Templates zu verwenden und nutzt eine sehr Python-nahe Syntax für Schleifen und
andere Kontrollstrukturen. Beide Features sind jedoch eher als Nachteile zu bewerten. Ersteres führt
gerade bei weniger erfahrenen Entwicklern leicht dazu, dass zuviel Logik in ein Template integriert
wird \enquote{weil es einfacher ist}. Die Python-nahe Syntax wäre eigentlich ein Vorteil, wenn sie
wie in Jinja2 oder Django so umgesetzt wäre, dass sie gleichzeitig entwickler- und vor allem
designerfreundlich ist. Leider muss selbst in Templates der Doppelpunkt zum Beginn eines neuen
Blocks eingefügt werden, was leicht zu Fehlern führt und keinerlei Nutzen hat - in Python selbst
dient es dazu, den Anfang eines Blocks deutlicher zu kennzeichnen und auch die Grammatik der Sprache
einfacher gestalten zu können. Dies ist jedoch in einem Template, in dem einzelne Logikausdrücke
bereits durch eine speziellen Syntax abgegrenzt, ist überflüssig.

Neben diesen beiden Frameworks existieren noch viele weitere Alternativen - das
\href{http://wiki.python.org/moin/WebFrameworks}{Python-Wiki}\footnote{\citep{pythonwebframeworks}}
enthält eine ausführliche
Framework-Liste einschließlich einer sehr kurzen Beschreibung eines jeden Frameworks, sodass man
sich leicht einen Überblick verschaffen kann, ob ein bestimmtes Framework für einen bestimmten
Use-Case geeignet sein könnte oder nicht.

\section{Fazit}
Flask bietet sowohl Entwicklern, die erst wenig Erfahrung mit Python oder Webanwendungen haben,
als auch erfahrenen Python-Entwicklern, die im Handumdrehen ihren eigenen WSGI-Server schreiben
könnten, eine komfortable Möglichkeit, Webanwendungen schnell und sauber zu entwickeln.

Dadurch, dass Flask-Anwendungen aus einer einzigen Datei bestehen können, kann man gerade kleinere
Anwendungen sehr einfach damit entwickeln. Die Nutzung von Decorators ermöglicht es, Funktionen
einfach so zu schreiben, dass sie die gewünschte Aufgabe erfüllen, und dann mit einer einzigen
zusätzlichen Zeile Code mit einer URL zu verknüpfen.

Aber auch größere Anwendungen sind problemlos möglich - man kann beim Routing auch komplett auf
Decorators verzichten und somit alle URLs an einer zentralen Stelle definieren. Dadurch und durch
die Modularisierung der Anwendung mittels Blueprints bleibt auch eine große Flask-Anwendung gut
wartbar und erlaubt es einem neuen Entwickler, sich schnell einzuarbeiten.

Auch ein bestehendes Projekt, welches beispielsweise CGI nutzt, kann somit unter Umständen sehr
einfach zu Flask migriert werden indem man den Code einer jeden Datei in eine Funktion verpackt und
mit dem entsprechenden Routing-Decorator versieht. Selbstverständlich ist danach noch Refactoring
notwendig, um beispielsweise \emph{duplicate code} zu entfernen und die Vorteile von Flask zu nutzen
- insbesondere beispielsweise das Erzeugen von URLs anhand der dazugehörigen Funktion.

Kurz gesagt, Flask ist ein komfortables Framework und wenn man Python nutzen will und nicht bereits
viel Erfahrung mit einem anderen Framework wie Django besitzt ist es auf jeden Fall einen Blick
wert.
