\chapter{Ausblick}

\section{Verbreitung}
Flask wird in einer Vielzahl von Projekten eingesetzt. Während es sich dabei größtenteils um private
Websites oder kleinere Open-Source-Projekte handelt wird Flask auch von einigen Firmen eingesetzt.

Die bekannteste Firma, die Flask einsetzt, dürfte \emph{DISQUS} sein. Bei \emph{disqus.com} handelt
es sich um einen Onlinedienst, der es Websitebetreibern ermöglicht, ohne großen Aufwand eine
Kommentarfunktion mit verschachtelten Diskussionen zu integrieren. Derzeit wird Disqus laut Anbieter
von ca. 91 Millionen Benutzern verwendet und ist auf ca. 1.8 Millionen Websites integriert, bei
denen man auch Größen wie CNN und Bloomberg findet.

\img{disqus-logo-dark.png}{100px}{Disqus-Logo}{Disqus-Logo}

Selbstverständlich wird Flask noch auf deutlich mehr Seiten eingesetzt - da bei der Benutzung von
Flask jedoch kein \enquote{Powered by Flask}-artiger Hinweis eingebaut werden muss, ist es in der
Regel für einen Außenstehenden nicht zu erkennen, dass eine Website Flask-basiert ist. Es gibt
jedoch auf der Flask-Website eine \href{http://flask.pocoo.org/community/poweredby/}{Liste}, in der
sich jeder Benutzer von Flask eintragen kann. Diese ist auch für angehende Flask-Entwickler
nützlich, da oftmals auch der Quellcode der Seiten verlinkt ist und man somit bereits vorhandenen
Flask-Code anschauen kann.

\section{Alternativen}
\todotext{Alternative Frameworks; kleinere und was großes (Django)}

\section{Fazit}
\todotext{Nun ja, ein Fazit. Was sonst ;)}
