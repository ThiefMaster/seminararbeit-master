\chapter{Flask}
\section{Allgemeines}

Bei Flask\footnote{\href{http://flask.pocoo.org}{http://flask.pocoo.org}} handelt es sich um ein
Python-basiertes Microframework für Webapplikationen. Das \emph{Micro} bedeutet, dass es sich bei
Flask um ein Framework handelt, welches nur die grundlegendsten Funktionen zur Entwicklung einer
Webanwendung zur Verfügung stellt und dem Entwickler somit ein solides Grundgerüst bietet ohne ihm
vorzugeben, welche Datenbank bzw. welche Datenbankabstraktion er zu benutzen hat.

Das Framework basiert auf der \emph{WSGI utility library}
Werkzeug\footnote{\href{http://werkzeug.pocoo.org}{http://werkzeug.pocoo.org}} welche
Lowlevel-Funktionen für das WSGI-Interface, URL-Routing und diverse HTTP-Features wie Header,
Cookies, Weiterleitungen und Formulardaten zur Verfügung stellt. In der Regel kommt man mit diesen
jedoch nicht direkt in Berührung, da Flask eigene APIs für häufig genutzte Funktionen bereitstellt.

\section{Aufbau}

Man kann Flask grob in die folgenden Bereiche aufteilen: \emph{Application}, \emph{Blueprints},
\emph{Routing}, \emph{Sessions}, \emph{Templates} und \emph{Signals}.

Die \emph{Application} ist eine Instanz der \lstinline{Flask}-Klasse und ist das zentrale Objekt der
Webanwendung. Sie implementiert dabei die WSGI-Applikation und alle applikationsspezifischen
Operationen laufen über diese Instanz.

Bei \emph{Blueprints} handelt es sich vereinfacht um Objekte die über eine ähnliche API wie das
\lstinline{Flask}-Objekt verfügen. Sie ermöglichen es, Teile einer Applikation (beispielsweise die
Benutzerverwaltung) wiederverwendbar zu gestalten und mit einem einzigen Funktionsaufruf alle im
\emph{Blueprint} gespeicherten URL-Handler und Templates mit einer konkreten Applikation
(repräsentiert durch eine \lstinline{Flask}-Instanz) zu verknüpfen.

Das \emph{Routing} ist dafür zuständig, URLs der Form \emph{/foo/bar/} mit Python-Funktionen oder
-Klassen zu verknüpfen. Dabei können die einzelnen Pfadelemente auch dynamisch sein, sodass saubere
URLs ermöglicht werden. Beispielsweise könnte die Routendefiniton für einen Artikel in einem
Wiki der String \lstinline{'/wiki/<name>'} sein. In diesem Fall würde die verknüpfte Funktion für
jede mit \emph{/wiki/} beginnende URL ausgeführt werden. Ebenfalls vom Routing-System unterstützt
wird die umgekehrte Richtung, d.h. das Generieren von URLs anhand des Namens der verknüpften
Python-Funktion. Dies erlaubt dem Entwickler, in seinen Templates keinerlei URLs hardcoden zu
müssen. Darüberhinaus wird vermieden, dass URLs mühsam mit Stringoperationen zusammengebaut werden
müssen - stattdessen werden einfach der Funktionsname und die jeweiligen Parameter an die Funktion
\lstinline{url_for()} übergeben.

Die in den meisten Webanwendungen dringend benötigten \emph{Sessions} sind in Flask durch signierte
Cookies realisiert. Dabei handelt es sich um ein Cookies, welche durch eine kryptographische
Signatur vor Manipulationen geschützt sind. Der Benutzer kann daher also den Inhalt der
Sessionvariablen ohne Weiteres auslesen, ist jedoch nicht in der Lage ihn zu verändern. Der
Entwickler kann auf die Session wie auf ein normales Python-\lstinline{dict} zugreifen; sollte die
Signature eines Wertes ungültig sein, wird dieser beim Parsen des Session-Cookies ignoriert und
taucht dementsprechend nicht in der Session auf.

\emph{Templates} sind aus einer modernen Webanwendung kaum wegzudenken. Während früher -
insbesondere zu den Anfangszeiten von PHP und Perl - oftmals HTML-Code direkt in die
Quellcodedateien geschrieben wurde, ist diese Vermischung von Layout/Design und Logik heute
undenkbar und würde sowohl die Lesbarkeit als auch die Wartbarkeit stark senken. Daher integriert
Flask standardmäßig die Templateengine
Jinja2\footnote{\href{http://jinja.pocoo.org}{http://jinja.pocoo.org}}. Dabei handelt es sich um
eine komfortable Templateengine, die Templates \emph{just in time} in Python-Bytecode kompiliert und
dadurch eine hohe Performance bietet. Die Templatesyntax kann angepasst werden um unabhängig vom
Ausgabeformat die Lesbarkeit der Templates zu erhalten - beispielsweise ist die Standardsyntax, die
\lstinline|| und \lstinline|{{ ... }}| nutzt für ein \LaTeX-Dokument nicht zu
gebrauchen.

\emph{Signals} ermöglichen es, Code auszuführen, wenn framework-intern bestimmte Ereignisse
eintreten. Beispielsweise wird das \lstinline{request_tearing_down}-Signal immer dann ausgelöst,
wenn ein HTTP-Request vollständig bearbeitet wurde. Da diese Signal immer ausgelöst wird - auch wenn
eine Exception aufgetreten ist - bietet es sich beispielsweise dazu an, eine Datenbankverbindung zu
schließen oder sonstige Aufräumarbeiten auszuführen.
