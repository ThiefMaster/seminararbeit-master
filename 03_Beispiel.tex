\chapter{Flask - Praktische Nutzung}

Die Hochschule Karlsruhe bietet auf ihrer Website eine Onlineversion des \enquote{Schwarzen Bretts}.
Dabei handelt es sich um eine reine HTML-Seite, die zwar im Browser ohne Weiteres angezeigt werden
kann, jedoch mangels einer RSS/Atom-Version nicht mit Feed-Readern kompatibel ist. Da ein Newsfeed
bei einer solchen Seite jedoch angebracht ist und den Komfort stark erhöht, war die einfachste
Lösung die Entwicklung einer externen Seite, die die HTML-Seite ausliest und in einen Atom-Feed
konvertiert.

\section{Anforderungen}

Obwohl es sich um eine sehr einfache Anwendung handelt gibt es diverse Anforderungen die erfüllt
sein sollten:

\begin{itemize}
\item Da es äußerst ineffizient wäre, die HTML-Seite bei jedem Aufruf des Feeds erneut abzurufen,
muss diese für eine gewisse Zeit gecacht werden. Neben den kürzeren Ladezeiten des Feeds verringert
dies ebenfalls die Last des Hochschulservers, da keine unnötigen Anfragen gesendet werden. Da das
schwarze Brett in der Regel nicht sehr oft aktualisiert wird und keine kurzfristigen Ankündigungen
enthält bietet sich eine Cache-Dauer im niedrigen Stundenbereich an.

\item Die Applikation greift auf externe Ressourcen zu, deren Adresse sich theoretisch jederzeit
ändern kann. Externe URLs sollten daher nicht direkt im Code stehen sondern konfigurierbar sein

\item Um den Code nicht übermäßig zu verkomplizieren soll die HTML-Seite mit der Python-Bibliothek
\emph{BeautifulSoup} geparst werden statt beispielsweise eine XSL-Transformation zu verwenden. Dies
führt zwar dazu, dass bei einer Änderung an der HTML-Struktur Codeänderungen notwendig werden,
allerdings kann dieser Nachteil vernachlässigt werden.

\item Es sollen mehrere Feeds unterstützt werden. Dies ist notwendig da es auch mehrere Schwarze
Bretter gibt. Der Einfachheit halber werden nur die Neuigkeiten des Fachgebiets Informatik in Feeds
konvertiert, jedoch ist eine Erweiterung auf andere Fachgebiete durch die konfigurierbaren URLs
prinzipiell möglich.

\item Es sollen saubere URLs verwendet werden und auf GET-Parameter nach Möglichkeit verzichtet
werden. Da der einzige dynamische Parameter der Name des Feeds ist, bietet sich eine URL nach dem
Schema \emph{/<name>.atom} an. Während Dateierweiterungen oftmals im Widerspruch zu \emph{Clean
URLs} stehen sind sie in diesem Fall zulässig und auch sinnvoll, da somit sofort klar ist, dass sich
hinter der URL ein Atom-Feed verbirgt.

\item Neben den Feeds soll es eine HTML-Seite geben auf der die verfügbaren Feeds verlinkt sind. Der
dazugehörige HTML-Code soll in eine Template ausgelagert werden.
\end{itemize}

\section{Implementierung als Flask-Anwendung}

\subsection{Externe Module und Flask-Extensions}

Neben dem in den Anforderungen bereits erwähnten \emph{BeautifulSoup} zum Parsen der
HTML-Seiten bietet sich für das Caching die Flask-Erweiterung \emph{Flask-Cache} an. Diese bietet
eine einfache Abstraktion verschiedener Cache-Backends (Dateisystem, Memcached, Redis, \ldots).
Durch diese Abstraktion können erweiterte Features von Redis oder Memcached nicht genutzt werden;
dies ist jedoch in diesem Fall unproblematisch da diese nicht benötigt werden.

\subsection{Initialisierung und Konfiguration}

\begin{lstlisting}[caption=HSKAtom - Initialisierung,label=lst:hskatom-init]
# Initialize application
app = Flask(__name__)
app.config.from_pyfile('hskatom.cfg')
cache = Cache(app)

# Config vars
NEWS_URLS = app.config.setdefault('NEWS_URLS', {})
FEED_TITLE = app.config.setdefault('FEED_TITLE',
    'Hochschule Karlsruhe - News')
NEWS_CACHE_DURATION = app.config.setdefault(
    'NEWS_CACHE_DURATION', 3600)
\end{lstlisting}

In \autoref{lst:hskatom-init} wird zunächst eine Flask-Anwendung erstellt. Flask bietet verschiedene
Möglichkeiten, konfigurierbare Einstellungen zu laden. In diesem Fall werden die Einstellungen mit
der Methode \lstinline{app.config.from_pyfile()} aus einer Python-Datei imporiert. Alle dort
definierten globalen Variablen, die ausschließlich aus Großbuchstaben und Unterstrichen bestehen,
werden danach über das unter dem Namen \lstinline{app.config} erreichbare \lstinline{dict}
zugänglich gemacht. Während es in diesem Fall weder notwendig noch angebracht ist, Python-Code in
der Konfigurationsdatei zu verwenden, der über einfache Variablenzuweisungen hinausgeht, hat die
Möglichkeit beliebigen Code zu benutzen diverse Vorteile. Beispielsweise ermöglicht dies, einzelne
Werte aus anderen zusammenzusetzen oder Datumswerte direkt als \lstinline{datetime}-Objekt
auszudrücken.

Nachdem die Konfiguration geladen wurde, wird das Cache-Interface erstellt - dabei handelt es sich
um eine Instanz der Klasse \lstinline{Cache}. Diese befindet sich in dem Modul
\lstinline{flask.ext.cache}. Bei dem Modul \lstinline{flask.ext} handelt es sich um einen Proxy, der
automatisch das zu der Extension gehörende Modul bzw. Package importiert. Dabei werden sowohl neue
Module unterstützt, die unter dem globalen Namen \lstinline{flask_XYZ} verfügbar sind als auch alte
Module, die sich im Namespace \lstinline{flaskext} befinden und somit unter dem Namen
\lstinline{flaskext.XYZ} verfügbar sind.

Der Einfachheit halber werden die applikationsspezifischen Konfigurationsoptionen in globalen
Variablen abgelegt; die Nutzung der \lstinline{setdefault()}-Methode garantiert jedoch zusätzlich,
dass sie auch in \lstinline{app.config} definiert sind und sinnvolle Defaultwerte besitzen die
zumindest nicht zu einem Fehler führen wenn ihnen in der Konfigurationsdatei kein Wert zugewiesen
wurde.
