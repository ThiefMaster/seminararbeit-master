\chapter{Flask - Praktische Nutzung}

Die Hochschule Karlsruhe bietet auf ihrer Website eine Onlineversion des \enquote{Schwarzen Bretts}.
Dabei handelt es sich um eine reine HTML-Seite, die zwar im Browser ohne Weiteres angezeigt werden
kann, jedoch mangels einer RSS/Atom-Version nicht mit Feed-Readern kompatibel ist. Da ein Newsfeed
bei einer solchen Seite jedoch angebracht ist und den Komfort stark erhöht, war die einfachste
Lösung die Entwicklung einer externen Seite, die die HTML-Seite ausliest und in einen Atom-Feed
konvertiert.

\section{Anforderungen}

Obwohl es sich um eine sehr einfache Anwendung handelt gibt es diverse Anforderungen die erfüllt
sein sollten:

\begin{itemize}
\item Da es äußerst ineffizient wäre, die HTML-Seite bei jedem Aufruf des Feeds erneut abzurufen,
muss diese für eine gewisse Zeit gecacht werden. Neben den kürzeren Ladezeiten des Feeds verringert
dies ebenfalls die Last des Hochschulservers, da keine unnötigen Anfragen gesendet werden. Da das
schwarze Brett in der Regel nicht sehr oft aktualisiert wird und keine kurzfristigen Ankündigungen
enthält bietet sich eine Cache-Dauer im niedrigen Stundenbereich an.

\item Die Applikation greift auf externe Ressourcen zu, deren Adresse sich theoretisch jederzeit
ändern kann. Externe URLs sollten daher nicht direkt im Code stehen sondern konfigurierbar sein

\item Um den Code nicht übermäßig zu verkomplizieren soll die HTML-Seite mit der Python-Bibliothek
\emph{BeautifulSoup} geparst werden statt beispielsweise eine XSL-Transformation zu verwenden. Dies
führt zwar dazu, dass bei einer Änderung an der HTML-Struktur Codeänderungen notwendig werden,
allerdings kann dieser Nachteil vernachlässigt werden.

\item Es sollen mehrere Feeds unterstützt werden. Dies ist notwendig da es auch mehrere Schwarze
Bretter gibt. Der Einfachheit halber werden nur die Neuigkeiten des Fachgebiets Informatik in Feeds
konvertiert, jedoch ist eine Erweiterung auf andere Fachgebiete durch die konfigurierbaren URLs
prinzipiell möglich.

\item Es sollen saubere URLs verwendet werden und auf GET-Parameter nach Möglichkeit verzichtet
werden. Da der einzige dynamische Parameter der Name des Feeds ist, bietet sich eine URL nach dem
Schema \emph{/<name>.atom} an. Während Dateierweiterungen oftmals im Widerspruch zu \emph{Clean
URLs} stehen sind sie in diesem Fall zulässig und auch sinnvoll, da somit sofort klar ist, dass sich
hinter der URL ein Atom-Feed verbirgt.

\item Neben den Feeds soll es eine HTML-Seite geben auf der die verfügbaren Feeds verlinkt sind. Der
dazugehörige HTML-Code soll in eine Template ausgelagert werden.
\end{itemize}

\section{Implementierung als Flask-Anwendung}

\subsection{Externe Module und Flask-Extensions}

Neben dem in den Anforderungen bereits erwähnten \emph{BeautifulSoup} zum Parsen der
HTML-Seiten bietet sich für das Caching die Flask-Erweiterung \emph{Flask-Cache} an. Diese bietet
eine einfache Abstraktion verschiedener Cache-Backends (Dateisystem, Memcached, Redis, \ldots).
Durch diese Abstraktion können erweiterte Features von Redis oder Memcached nicht genutzt werden;
dies ist jedoch in diesem Fall unproblematisch da diese nicht benötigt werden.

Werkzeug enthält bereits standardmäßig eine Klasse zum Erzeugen von Atom-Feeds, daher ist dazu kein
separates Modul notwendig.

\subsection{Initialisierung und Konfiguration}

\begin{lstlisting}[caption=HSKAtom - Initialisierung,label=lst:hskatom-init]
# Initialize application
app = Flask(__name__)
app.config.from_pyfile('hskatom.cfg')
cache = Cache(app)

# Config vars
NEWS_URLS = app.config.setdefault('NEWS_URLS', {})
FEED_TITLE = app.config.setdefault('FEED_TITLE',
    'Hochschule Karlsruhe - News')
NEWS_CACHE_DURATION = app.config.setdefault(
    'NEWS_CACHE_DURATION', 3600)
\end{lstlisting}

In \autoref{lst:hskatom-init} wird zunächst eine Flask-Anwendung erstellt. Flask bietet verschiedene
Möglichkeiten, konfigurierbare Einstellungen zu laden. In diesem Fall werden die Einstellungen mit
der Methode \lstinline{app.config.from_pyfile()} aus einer Python-Datei imporiert. Alle dort
definierten globalen Variablen, die ausschließlich aus Großbuchstaben und Unterstrichen bestehen,
werden danach über das unter dem Namen \lstinline{app.config} erreichbare \lstinline{dict}
zugänglich gemacht. Während es in diesem Fall weder notwendig noch angebracht ist, Python-Code in
der Konfigurationsdatei zu verwenden, der über einfache Variablenzuweisungen hinausgeht, hat die
Möglichkeit beliebigen Code zu benutzen diverse Vorteile. Beispielsweise ermöglicht dies, einzelne
Werte aus anderen zusammenzusetzen oder Datumswerte direkt als \lstinline{datetime}-Objekt
auszudrücken.

Nachdem die Konfiguration geladen wurde, wird das Cache-Interface erstellt - dabei handelt es sich
um eine Instanz der Klasse \lstinline{Cache}. Diese befindet sich in dem Modul
\lstinline{flask.ext.cache}. Bei dem Modul \lstinline{flask.ext} handelt es sich um einen Proxy, der
automatisch das zu der Extension gehörende Modul bzw. Package importiert. Dabei werden sowohl neue
Module unterstützt, die unter dem globalen Namen \lstinline{flask_XYZ} verfügbar sind als auch alte
Module, die sich im Namespace \lstinline{flaskext} befinden und somit unter dem Namen
\lstinline{flaskext.XYZ} verfügbar sind.

Der Einfachheit halber werden die applikationsspezifischen Konfigurationsoptionen in globalen
Variablen abgelegt; die Nutzung der \lstinline{setdefault()}-Methode garantiert jedoch zusätzlich,
dass sie auch in \lstinline{app.config} definiert sind und sinnvolle Defaultwerte besitzen die
zumindest nicht zu einem Fehler führen wenn ihnen in der Konfigurationsdatei kein Wert zugewiesen
wurde.

\subsection{Routing}

Die Feed-Übersicht ist die Startseite der Anwendung, daher ist sie über \emph{/} zu erreichen; der
dazugehörige \lstinline{app.route()}-Aufruf ist selbsterklärend und entspricht exakt dem Beispiel in
\autoref{lst:simple-flask-app}.

\begin{lstlisting}[caption=HSKAtom - Routing,label=lst:hskatom-route-index]
@app.route('/')
def index():
    """show feed list"""
\end{lstlisting}

Für die jeweiligen Feeds sollen jedoch nicht beliebige Namen akzeptiert werden sondern nur solche,
zu denen auch eine URL konfiguriert ist. Eine Möglichkeit, die zu tun, wäre ein einfaches
\lstinline{if}-Statement und ein Aufruf von \lstinline{abort(404)} im Falle eines ungültigen Namens.
Schöner ist es jedoch, das Routing-System entsprechend zu konfigurieren, sodass bereits frühzeitig
nur gültige Werte zugelassen werden. Dies ist mit dem in \autoref{lst:hskatom-route-atom} gezeigten
\emph{argument converter} realisiert. Dazu wird eine neue Klasse erstellt, die von der im
Werkzeug-Routing-System definierten Klasse \lstinline{BeseConverter} erbt. In dieser wird der Regex
\lstinline{'[^/]+'}, der ein beliebiges Pfadsegement (also ein String, der keinen Slash enthält)
matchen würde, durch einen alternativen Regex ersetzt, der genau die Schlüssel des
\lstinline{NEWS_URLS}-Dicts matcht. Durch das Hinzufügen der neuen Klasse zum
\lstinline{app.url_map.converters}-Dict steht diese im Routing-System zur Verfügung und kann in einer
Routendefinition genutzt werden; die Syntax dazu ist \lstinline{'<converter:varname>'}.

\begin{lstlisting}[caption=HSKAtom - Routing,label=lst:hskatom-route-atom]
class NewsKeyConverter(BaseConverter):
    """Matches a news key"""

    def __init__(self, map):
        BaseConverter.__init__(self, map)
        self.regex = '(?:%s)' % '|'.join(re.escape(x) for x in NEWS_URLS)
app.url_map.converters['newskey'] = NewsKeyConverter

@app.route('/<newskey:key>.atom')
def atom(key):
    """create the atom feed"""
\end{lstlisting}

\subsection{Erzeugen des Atom-Feeds}

Beim Atom-Format handelt es sich um einen in \footnote{\href{http://www.ietf.org/rfc/rfc4287}{RFC
4287} \citep{rfc4287}} beschriebenen Standard zur \emph{Content Syndication} der von den meisten verbreiteten
Browsern und Standalone-Feed-Readern unterstützt wird. Die Funktion \lstinline{retrieve_news()} ruft
die Ankündigungen von dem entsprechenden Schwarzen Brett ab und gibt sie als Python-Datenstruktur
zurück. Da ein Atom-Feed zwingend einen Last-Updated-Timestamp benötigt gibt die Funktion neben den
eigentlichen Daten auch einen solchen Timestamp zurück. Da die Einträge auf dem Schwarzen Breet
jedoch nicht mit einem Datum versehen sind und dort sowieso nur aktuell relevante Einträge angezeigt
werden, wird den einzelnen Einträgen ein Dummy-Datum, der 1.1.1970, zugewiesen. Schöner wäre es
natürlich die Schwarzen Bretter regelmäßig auf neue Einträge zu überprüfen und das Datum, an dem ein
Eintrag dazukommt, zu verwenden. Dazu wären jedoch Cronjobs oder regelmäßige Zugriffe auf die
Anwendung notwendig. Da jeder Eintrag auf der Hochschulseite eine eindeutige ID hat kann diese
sowohl als eindeutige ID des Feed-Eintrags benutzt werden als auch um jeden Eintrag mit einer URL zu
versehen, die zum Original auf der Hochschulseite führt.

\begin{lstlisting}[caption=HSKAtom - Atom-Feed,label=lst:hskatom-atom]
def atom(key):
    items, last_update = retrieve_news(key)
    url = NEWS_URLS[key]
    feed = AtomFeed(FEED_TITLE, feed_url=request.url,
        url=url, subtitle=key, updated=last_update)
    ts = datetime(1970, 1, 1)
    for item in items:
        item_url = '%s#%s' % (url, item['id'])
        feed.add(item['title'], item['content'],
            content_type='html', id=item_url,
            updated=ts, url=item_url)
    return feed.get_response()
\end{lstlisting}

Die Methode \lstinline{get_response()} des \lstinline{AtomFeed}-Objekts gibt ein
\lstinline{Response}-Objekt zurück, welches neben dem XML-Code des Feeds auch den dazugehörigen
Content-Type \emph{application/atom+xml} enthält, sodass ein Browser die Daten korrekt verarbeiten
kann. Im Firefox wird beispielsweise eine Vorschau der Einträge und ein Interface zum abonnieren des
Feeds angezeigt.

\img{hskatom-firefox.png}{400px}{Atom-Feed-Vorschau im Firefox}{Atom-Feed-Vorschau im Firefox}

\subsection{Abruf der Daten}

Um die Daten des Schwarzen Bretts in ein brauchbares Format zu konvertieren müssen sie zunächst vom
Hochschulserver abgerufen werden. Die einfachste Möglichkeit dazu ist das \emph{urllib}-Modul aus
der Python-Standardbibliothek. Aus dem HTML-Code lässt sich mit \emph{BeautifulSoup} ein DOM-Tree
aufbauen, der danach mit regulären Python-Kontrollstrukturen wie Schleifen durchlaufen werden kann.
Die relevanten Inhalte der Seite befinden sich innerhalb von
\lstinline[language=HTML]{<div id="content">}. Dieses Element enthält mehrere DIV-Elemente mit
\lstinline[language=HTML]{class="csc-default"} und einer einzigartigen ID. Titel und Text der
einzelnen Einträge sind ebenfalls voneinander getrennt, sodass auch diese problemlos geparst werden
können. Der Titel befindet sich in einem H4-Element während der Text in einem DIV mit
\lstinline[language=HTML]{class="last"} zu finden ist, welches wiederum ein Element mit
\lstinline[language=HTML]{class="content"} enthält. Da der Body eines Eintrags HTML enthalten kann,
kann dieser nicht wie der Titel einfach über das \lstinline{text}-Attribut ausgelesen werden. Das
Attribute \lstinline{contents} enthält jedoch alle Kindelemente, die unabhängig vom Typ in einen
String konvertiert werden können. Dies geschrieht durch Anwendung der Funktion \lstinline{unicode}
auf alle Elemente der \lstinline{contents}-Liste durch die Funktion
\lstinline{map(callable, iterable)}. Die einzelnen Elemente der Liste werden mit der
\lstinline{join()}-Funktion aneinandergehängt.

\subsection{Caching}

Um den Rückgabewert von \lstinline{retrieve_news()} zu cachen bietet \emph{Flask-Cache} den
Decorator \lstinline{cache.memoize(duration)}. Dieser cacht den Rückgabewert der damit dekorierten
Funktion für die angegebene Zeit. Als Cache-Key wird dabei eine Kombination aus dem Funktionsnamen
und den übergebenen Argumenten verwendet. Der Code, der die Funktion aufruft, kann sie wie bisher
nutzen - das Caching geschieht vollkommen transparent. Ohne Caching hätte es sich jedoch angeboten,
den Timestamp des Feeds in der \lstinline{atom()}-Funktion zu erzeugen und nicht ebenfalls von
\lstinline{retrieve_news()} zurückgeben zu lassen. Da sich der gecachte Inhalt jedoch nie verändert,
solange er im Cache liegt, ist es angebracht, auch den Timestamp des Feeds nur dann zu ändern, wenn
auch potentiell neue Daten vorliegen.


\begin{lstlisting}[caption=HSKAtom - Cache,label=lst:hskatom-cache]
@cache.memoize(NEWS_CACHE_DURATION)
def retrieve_news(key):
    """retrieve news from the hska site"""
\end{lstlisting}

In dieser Anwendung ist es in der Regel nicht notwendig, den Cache zu löschen, weshalb diese Option
nicht in die Anwendung integriert wurde. Sollte es jedoch aus irgendeinem Grund notwendig sein,
bietet \emph{Flask-Cache} dazu mit der Methode \lstinline{cache.delete_memoized(func, ...)}
die Möglichkeit, die gecachten Rückgabewerte der einer Funktion zu löschen, wobei optional auch die
Parameter angegeben werden können um nicht den gesamten Cache der Funktion löschen zu müssen.
Letzteres würde sich beispielsweise anbieten, wenn man nur den Cache eines bestimmten Feeds löschen
wollte.

\subsection{Die fertige Flask-Anwendung}

\autoref{lst:hskatom} enthält den vollständigen Code der \emph{HSKAtom}-Flask-Anwendung. Mit gerade
einmal 75 Zeilen Code, von denen gut ein Drittel auf das Parsen des HTML-Codes entfallen ist sie ein
gutes Beispiel dafür, wie kompakt eine Webanwendung mit Flask sein kann. Selbstverständlich ist der
Code wenn man Framework und externe Module dazurechnet deutlich länger als beispielsweise PHP-Code
der komplett auf externen Code verzichtet - dies ist allerdings kein Nachteil sondern zeugt von
gutem Programmierstil (möglichst viel wiederverwendbaren Code nutzen) und anders als PHP bietet
Python mit \emph{Virtualenv} und \emph{pip} eine komfortable Möglichkeit, Abhängigkeiten automatisch
zu installieren.

\lstinputlisting[caption=HSKAtom-Applikation,label=lst:hskatom]{code/hskatom.py}
